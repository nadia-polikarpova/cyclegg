\documentclass{article}[11pt]

\usepackage{amsmath}
\usepackage{amsfonts}

\newcommand{\cyclegg}{\textsc{Cyclegg}}
\newcommand{\add}[2]{\texttt{add}\ #1 \ #2}
\newcommand{\next}[1]{S \, #1}

\title{``Proof'' of equivalence between \cyclegg{} and equality-saturation-based proof using regular induction \\
  \begin{large} In the context of equational reasoning \end{large}}

\begin{document}

\maketitle{}

\section{Proof}

We will prove equivalence between these two forms by showing that having a
cyclic proof implies that there exists a proof only using the induction
hypothesis (hereafter to be referred to as the IH). It follows that this implies
there exists a cyclic proof if and only if there exists an ``uncycled'' proof
using only the IH.

\subsection{Proofs in \cyclegg{}}

\subsubsection{General form}

Proofs in \cyclegg{} may involve case splitting, but whenever we perform a case
split (or if-then-else split) we create a list of sub-proofs on specializations
of the original proof which combined imply the original proof. So we will
consider an arbitrary proof of the equality between two terms: the LHS (for
``left-hand side'') and RHS (for ``right-hand side'') with respect to some set
of universally quantified variables \(V\). By reasoning in this general way, we
can demonstrate how to uncycle an arbitrary sub-proof; if we know how to uncycle
an arbitrary sub-proof, we can uncycle the whole proof, since the whole proof is
just a list of sub-proofs for each case.

\subsubsection{Rewrites}

A sub-proof in \cyclegg{} consists of symbolic rewrites from the LHS to RHS.
Some of these rewrites are axioms created from the definitions of functions. For
example, we take it as an axiom that \(\forall n. \add Z n = n\). The other
rewrites are lemmas that we generate before we case split.

Before we case split, we look through each concrete term \(t_{lhs}\) that is
equal to the LHS and each concrete term \(t_{rhs}\) that is equal to the RHS.
Let \(\overline{v} \subseteq V\) be the variables shared between the two
expressions.\footnote{Since none of our rewrites may introduce new variables, it
  follows that \(t_{lhs}\) and \(t_{rhs}\) cannot have variables outside of
  those in \(V\) but they may have fewer variables.} We posit as a lemma that
\(\forall \overline{x}. t_{lhs}[\overline{v} \mapsto \overline{x}] = t_{rhs}[\overline{v} \mapsto \overline{x}]\)
where \(\overline{x}\) is a list of fresh variables created for each
\(\overline{v}\). Intuitively, this states you can instaniate the lemma
\(t_{lhs} = t_{rhs}\) with terms that replace \(overline{v}\).

This quantification isn't \emph{exactly} universal: we have a requirement that
the lemma is invoked with instantiaions of variables in \(\overline{v}\) which
are strictly structurally decreasing in at least one variable and not increasing
in any other. The important point here is that this guarantees our use of the
cyclic lemmas is logically sound.

For example, suppose we have \(t_{lhs} = \add Z n\) and \(t_{rhs} = n\). The
lemma we create from this is \(\forall x. \add Z x = x\), where instantiations
of \(x\) must be structurally smaller than the symbolic variable \(n\).

To summarize, a proof that LHS = RHS is a sequence of rewrites which are
either axioms or lemmas. We may justify each rewrite by instantiating its
universally quantified variables to match the sub-term that we are rewriting.

\subsection{Cyclic lemma unrolling}

Suppose now that we are using a cyclic lemma as part of the sub-proof that some
specialization of the original LHS = some specialization of the original RHS. I
claim that we can translate this sub-proof to use the IH instead.

Let us consider how the lemma was generated. At some point prior we were trying
to prove some specialization LHS' = RHS' in which some \(t_{lhs}'\) and
\(t_{rhs}'\) were used to construct the lemma. We have a series of rewrites that
justifies the equality between LHS' and \(t_{lhs}'\) as well as between RHS' and
\(t_{rhs}'\) with respect to some set of universally quantified variables
\(V'\).

Suppose that this cyclic lemma is used to justify the equivalence between terms
\(t_{1}\) and \(t_{2}\). In this case, we know that that \(t_{1}\) is an
instantiation of \(t_{lhs}'\) and \(t_{2}\) is an instantiation of \(t_{rhs}'\).

We know that we can rewrite \(t_{lhs}'\) to LHS'. And if we can rewrite LHS' to
RHS', we can then rewrite RHS' to \(t_{rhs}'\).\footnote{In essence, supposing
  that we can rewrite LHS' to RHS' justifies the cyclic lemma.} Our idea is to
apply these same rewrites to \(t_{1}\) and argue that with the IH we can rewrite
it to \(t_{2}\).

Consider the rewrites justifying the equivalence between \(t_{lhs}'\) and LHS'.
It is clear that for the rewrites which are axioms, we may also use them on
\(t_{1}\) since they are universally quantified. However, we need to justify
that the lemmas used on \(t_{lhs}'\) are usable on \(t_{1}\). This justification
is simple: because we require any instantiation of the lemma to be on variables
that are decreasing, the variables used in \(t_{1}\) must be valid for the same
lemmas for which the variables of \(t_{lhs}'\) are valid.

So we can rewrite \(t_{lhs}'\) so it looks like LHS'. Before we case split, we
could not bridge the gap between LHS' and RHS'. However, we know that the
rewritten \(t_{1}\) is over variables which are strictly decreasing with respect
to the variables in \(t_{lhs}'\). It, therefore, is strictly decreasing with
respect to the variables in the original LHS, because the variables in
\(t_{lhs}'\) all may not be bigger than the variables in the original LHS. So
the IH applies to the rewritten \(t_{1}\) and we may rewrite it again to look
like RHS'. From this point, we can use the same justification we made for
transforming \(t_{1}\) to look like \(t_{lhs}'\) to now bring it from the shape
of RHS' to \(t_{rhs}'\). And this is exactly \(t_{2}\), as desired.

\subsubsection{Case splits and if-then-else splits}

The part that seems most hand-wavy here is the claim that the original IH would
suffice even in a sub-proof. However, for case splits this make sense if you
think about it.

If your original proof is saying \(\forall n: \texttt{Nat} . x = y\), then its
sub-proofs will look like \(x[n \mapsto Z] = y[n \mapsto Z]\) and
\(\forall n': \texttt{Nat}. x[n \mapsto \next {n'}] = y[n \mapsto \next {n'}]\).
It follows that the IH subsumes both of these cases: for the first, we may
instantiate \(n = Z\) and for the second we may instantiate \(n = \next {n'}\).

So though we sometimes need to justify the equivalence
\(x[n \mapsto Z] = y[n \mapsto Z]\) using \(\forall n. x = y\), the latter will
always suffice in the context of case splits.

We may apply the same reasoning for if-then-else splitting, although there is a
caveat. Because an if-then-else split applies to an entire expression \(e\), we
need to be sure that it is actually possible for both the \texttt{true} and
\texttt{false} cases to arise from instantiations of variables in \(e\) - if
\(e\) is something like \texttt{x \&\& false}, we couldn't justify equivalence of
one of the if-then-else splits (the \texttt{true} case). However it's not clear
to me that cyclic lemmas would save us here. I think we have already made a
mistake if we're trying to justify the \texttt{true} case of an expression that
can never be true.

\subsubsection{Multiple lemmas}

This approach should work independently of how many lemmas there are since it
demonstrates constructively how to remove an individual lemma's invocation, but
does not affect any other parts of the proof.

\subsubsection{Mutually recursive lemmas}

What if a cyclic lemma \(\ell_{1}\) was used to justify the rewrites of another
\(\ell_{2}\) and \(\ell_{2}\) was used to justify rewrites in\(\ell_{1}\)?
Fortunately for us, I think this is impossible due to the fact that lemmas are
created from terms deemed equivalent with respect to only axioms and older
lemmas.

\section{Example}

As an example, let's consider proving that \(\add Z n = \add n Z\) with the standard
definition of the natural numbers

\[
\begin{array}{lcl}
  n &= & Z \\
    &| & \next n
\end{array}
\]

and the following axioms

\[
\begin{array}{ll}
  (1) & \add Z n = n \\
  (2) & \add {(\next m)} n = \next (\add m n)
\end{array}
\]

One way of proving this theorem is to utilize a cyclic lemma that \(n = \add n Z\).

This theorem requires a case split on \(n\): either \(n = Z\) or \(n = \next n'\). The
\(n = Z\) case follows immediately because \(\add Z Z = \add Z Z\). We may prove the
\(n = \next n'\) case as follows

\[
\begin{array}{lll}
  \add Z {(\next n')} & & \\
  &= \next n' & (1) \\
  &= \next (\add {n'} Z) & \text{lemma} \\
  &= \add {(\next n')} Z & (2) \\
\end{array}
\]

In order to propose the cyclic lemma \(n = \add n Z\), we know that \(n\) is in
the same equality class as \(\add Z n\) and \(\add n Z\) is in the same equality
class as \(\add n Z\). The latter is a trivial equality, but for the former we
have a proof that for all \(n\), \(n = \add Z n\).

We observe that the IH, though different in form, also demonstrates an equality
between the \emph{same} two equality classes.

So if we wanted to use the lemma to demonstrate that \(k = \add k Z\), we could
instead do the following:

\begin{enumerate}
  \item Rewrite \(k\) to \(\add Z k\) using our proof that for all \(n\), \(n = \add Z n\).
  \item Use the IH to rewrite \(\add Z k\) to \(\add k Z\).
  \item For an arbitrary cyclic unrolling, we might have to make another step
        from \(\add k Z\) to the desired term, but in this case the RHS of the
        lemma and IH are the same.
\end{enumerate}

Putting this to practice, we get the unrolled proof

\[
\begin{array}{lll}
  \add Z {(\next n')} & & \\
  &= \next n' & (1) \\
  &= \next (\add Z {n'}) & (1) \\
  &= \next (\add {n'} Z) & \text{IH} \\
  &= \add {(\next n')} Z & (2) \\
\end{array}
\]

\end{document}
